\documentclass[a4paper,12pt]{article}
\usepackage[utf8]{inputenc}
\usepackage[russian]{babel}
\usepackage{amsmath,amssymb,amsthm}
\usepackage{enumitem}
\usepackage[left=2cm,right=2cm,top=2cm,bottom=2cm]{geometry}
\usepackage{hyperref}
\usepackage{setspace} % Для управления межстрочными интервалами

% Стиль для теорем и определений без нумерации
\theoremstyle{definition}
\newtheorem*{definition}{Определение}
\newtheorem*{theorem}{Теорема}
\newtheorem*{proofnote}{Доказательство}
\renewenvironment{definition}{\textbf{Определение.}\par}{\par}
\renewenvironment{theorem}{\textbf{Теорема.}\par}{\par}
\renewenvironment{proofnote}{\textbf{Доказательство.}\par}{\qed}
% Удобные команды
\newcommand{\R}{\mathbb{R}}
\newcommand{\N}{\mathbb{N}}
\newcommand{\eps}{\varepsilon}
\newcommand{\limn}{\lim\limits_{n\to\infty}}
\newcommand{\limx}[1]{\lim\limits_{x\to #1}}

% Настройка отступов между абзацами
\setlength{\parskip}{0.5em} % Отступ между абзацами
\setlength{\parindent}{0pt} % Отступ первой строки абзаца

\begin{document}

% Титульный лист
\begin{titlepage}
\begin{center}
    \vspace*{5cm}
    
    {\Huge\textbf{Билеты для экзамена \\[0.3cm] по математическому анализу}}
    
    \vspace{1.5cm}
    
    {\Large 1 семестр}
    
    \vspace{1.5cm}
    
    {\LARGE\textbf{Искусственный интеллект и наука о данных}}
    
    \vspace{2cm}
    
    {\large На основе учебника \\ О.Л. Виноградова \\ «Математический анализ»}
    
    \vspace{2cm}
    
    \begin{minipage}{0.8\textwidth}
    \centering
    \rule{\textwidth}{1pt}
    
    \vspace{0.5cm}
    
    {\large Составители:} \\[0.3cm]
    {\large Deepseek, ChatGPT и Artemka}
    
    \vspace{0.5cm}
    
    \end{minipage}
    
    \vfill
\end{center}
\end{titlepage}

% Начинаем билеты с новой страницы
\newpage

\section*{Билет 1. Теорема о существовании $\sup E$}

\begin{definition}
Пусть $E \subset \mathbb{R}$, $E \neq \varnothing$, $E$ ограничено сверху. Число $b \in \mathbb{R}$ называется \textbf{точной верхней гранью} множества $E$ и обозначается $\sup E$, если:
\begin{enumerate}
    \item $\forall x \in E : x \leq b$ (т.е. $b$ — верхняя граница);
    \item $\forall \varepsilon > 0 \; \exists x \in E : x > b - \varepsilon$ (т.е. любое число меньшее $b$ не является верхней границей).
\end{enumerate}
\end{definition}

\begin{theorem}[существование $\sup E$]
Всякое непустое ограниченное сверху подмножество $\mathbb{R}$ имеет точную верхнюю грань.
\end{theorem}

\begin{proofnote}
Пусть $E \neq \varnothing$, $E$ ограничено сверху. Выберем $x_0 \in E$ и $M$ — верхнюю границу $E$ ($x_0 \leq M$). Положим $[a_1, b_1] = [x_0, M]$. Отрезок $[a_1, b_1]$ обладает свойствами:
\[
[a_1, b_1] \cap E \neq \varnothing, \quad (b_1, +\infty) \cap E = \varnothing.
\]

Рассмотрим середину отрезка $c_1 = \frac{a_1 + b_1}{2}$. Если $(c_1, b_1] \cap E \neq \varnothing$, то положим $[a_2, b_2] = [c_1, b_1]$, иначе $[a_2, b_2] = [a_1, c_1]$. В обоих случаях:
\[
[a_2, b_2] \cap E \neq \varnothing, \quad (b_2, +\infty) \cap E = \varnothing.
\]

Продолжая процесс, получим последовательность вложенных отрезков $\{[a_n, b_n]\}$, для которых:
\[
[a_n, b_n] \cap E \neq \varnothing, \quad (b_n, +\infty) \cap E = \varnothing, \quad b_n - a_n = \frac{M - x_0}{2^{n-1}} \to 0.
\]

По теореме о вложенных отрезках (аксиома Кантора) существует единственная точка $c$, принадлежащая всем отрезкам, причём $a_n \to c$, $b_n \to c$.

Покажем, что $c = \sup E$:
\begin{enumerate}
    \item Если $x \in E$, то $x \leq b_n$ для всех $n$ (иначе $x > b_n$ для некоторого $n$, что противоречит свойству $(b_n, +\infty) \cap E = \varnothing$). Переходя к пределу, получим $x \leq c$. Значит, $c$ — верхняя граница.
    \item Возьмём $\varepsilon > 0$. Так как $a_n \to c$, найдётся $N$ такой, что $a_N > c - \varepsilon$. По построению $[a_N, b_N] \cap E \neq \varnothing$, значит, существует $x \in E$, $x \geq a_N > c - \varepsilon$. Таким образом, $c - \varepsilon$ не является верхней границей.
\end{enumerate}

Следовательно, $c = \sup E$. 
\end{proofnote}

\section*{Билет 2. Теорема о существовании $\inf E$}

\begin{definition}
Пусть $E \subset \mathbb{R}$, $E \neq \varnothing$, $E$ ограничено снизу. Число $a \in \mathbb{R}$ называется \textbf{точной нижней гранью} множества $E$ и обозначается $\inf E$, если:
\begin{enumerate}
    \item $\forall x \in E : x \geq a$ (т.е. $a$ — нижняя граница);
    \item $\forall \varepsilon > 0 \; \exists x \in E : x < a + \varepsilon$ (т.е. любое число большее $a$ не является нижней границей).
\end{enumerate}
\end{definition}

\begin{theorem}[существование $\inf E$]
Всякое непустое ограниченное снизу подмножество $\mathbb{R}$ имеет точную нижнюю грань.
\end{theorem}

\begin{proofnote}
Пусть $E$ ограничено снизу. Рассмотрим множество $-E = \{-x : x \in E\}$. Оно ограничено сверху (так как $E$ ограничено снизу). По теореме о существовании $\sup E$ у него существует $\sup(-E) = s$. Положим $i = -s$. Докажем, что $i = \inf E$:

\begin{enumerate}
    \item Для любого $x \in E$ имеем $-x \in -E$, значит, $-x \leq s$, откуда $x \geq -s = i$. Следовательно, $i$ — нижняя граница $E$.
    \item Возьмём $\varepsilon > 0$. Так как $s = \sup(-E)$, существует $y \in -E$ такой, что $y > s - \varepsilon$. Пусть $y = -x$ для некоторого $x \in E$. Тогда $-x > s - \varepsilon$, т.е. $x < -s + \varepsilon = i + \varepsilon$. Значит, $i + \varepsilon$ не является нижней границей.
\end{enumerate}

Таким образом, $i = \inf E$.
\end{proofnote}

\section*{Билет 3. Предел последовательности: определение, свойства, единственность}

\begin{definition}
Число $a \in \mathbb{R}$ называется \textbf{пределом последовательности} $\{x_n\}$, если:
\[
\forall \varepsilon > 0 \; \exists N \in \mathbb{N} \; \forall n > N : |x_n - a| < \varepsilon.
\]
Обозначение: $\lim\limits_{n \to \infty} x_n = a$ или $x_n \to a$.
\end{definition}

\textbf{Свойства:}
\begin{enumerate}
    \item \textbf{Единственность предела.} Если $x_n \to a$ и $x_n \to b$, то $a = b$.
    \item \textbf{Ограниченность сходящейся последовательности.} Если $x_n \to a$, то $\exists M > 0 \; \forall n : |x_n| \leq M$.
    \item \textbf{Предельный переход в неравенстве.} Если $x_n \leq y_n$ для всех $n$ и существуют пределы $\lim x_n = a$, $\lim y_n = b$, то $a \leq b$.
    \item \textbf{Теорема о сжатой последовательности.} Если $x_n \leq y_n \leq z_n$ для всех $n$ и $\lim x_n = \lim z_n = a$, то $\lim y_n = a$.
\end{enumerate}

\begin{proofnote}[единственности]
Предположим, что $x_n \to a$ и $x_n \to b$, $a \neq b$. Пусть $\varepsilon = \frac{|a-b|}{2} > 0$. Тогда:
\[
\exists N_1 \; \forall n > N_1 : |x_n - a| < \varepsilon, \quad \exists N_2 \; \forall n > N_2 : |x_n - b| < \varepsilon.
\]
Возьмём $n > \max(N_1, N_2)$. Тогда:
\[
|a-b| \leq |a - x_n| + |x_n - b| < \varepsilon + \varepsilon = |a-b|,
\]
что противоречиво. Значит, $a = b$.
\end{proofnote}
\begin{proofnote}[ограниченности]
Пусть $x_n \to a$. По определению предела, для $\varepsilon = 1$ существует номер $N$ такой, что для всех $n > N$ выполняется:
\[
|x_n - a| < 1.
\]

Рассмотрим два случая:

1. Для $n > N$:
\[
|x_n| = |x_n - a + a| \leq |x_n - a| + |a| < 1 + |a|.
\]

2. Для $n \leq N$: конечное множество $\{x_1, x_2, \dots, x_N\}$ ограничено, так как оно конечно. Обозначим:
\[
M_1 = \max\{|x_1|, |x_2|, \dots, |x_N|\}.
\]
Тогда для всех $n \leq N$:
\[
|x_n| \leq M_1.
\]

Выберем:
\[
M = \max\{M_1, 1 + |a|\}.
\]

Тогда для всех $n \in \mathbb{N}$:
\[
|x_n| \leq M.
\]

Таким образом, последовательность $\{x_n\}$ ограничена.

\vspace{0.5cm}

\textbf{Альтернативное доказательство ограниченности:}

Поскольку $x_n \to a$, для $\varepsilon = 1$ найдётся $N$ такое, что для всех $n > N$:
\[
|x_n - a| < 1 \quad \Rightarrow \quad |x_n| < 1 + |a|.
\]

Пусть 
\[
M = \max\{|x_1|, |x_2|, \dots, |x_N|, 1 + |a|\}.
\]

Тогда для любого $n \in \mathbb{N}$:
\[
|x_n| \leq M.
\]

Следовательно, последовательность $\{x_n\}$ ограничена.
\end{proofnote}


\section*{Билет 4. $\lim c$, $\lim c x_n$, $\lim(x_n + y_n)$}

\begin{theorem}
\begin{enumerate} 
    \item $\lim\limits_{n \to \infty} c = c$ (постоянная последовательность).
    \item Если $\lim x_n = a$, то $\lim (c x_n) = c a$.
    \item Если $\lim x_n = a$, $\lim y_n = b$, то $\lim (x_n + y_n) = a + b$.
\end{enumerate}
\end{theorem}

\begin{proofnote}
\begin{enumerate}
    \item Для любого $\varepsilon > 0$ возьмём $N = 1$. Тогда для всех $n > N$ имеем $|c - c| = 0 < \varepsilon$.
    \item Пусть $x_n \to a$. Возьмём $\varepsilon > 0$. Если $c = 0$, то $c x_n = 0$, предел равен 0. Если $c \neq 0$, выберем $N$ такой, что $|x_n - a| < \frac{\varepsilon}{|c|}$ для всех $n > N$. Тогда:
    \[
    |c x_n - c a| = |c| |x_n - a| < |c| \cdot \frac{\varepsilon}{|c|} = \varepsilon.
    \]
    \item Пусть $x_n \to a$, $y_n \to b$. Для $\varepsilon > 0$ выберем $N_1$, $N_2$ такие, что:
    \[
    |x_n - a| < \frac{\varepsilon}{2} \; \forall n > N_1, \quad |y_n - b| < \frac{\varepsilon}{2} \; \forall n > N_2.
    \]
    Пусть $N = \max(N_1, N_2)$. Тогда для $n > N$:
    \[
    |(x_n + y_n) - (a + b)| \leq |x_n - a| + |y_n - b| < \frac{\varepsilon}{2} + \frac{\varepsilon}{2} = \varepsilon.
    \]
\end{enumerate}
\end{proofnote}

\section*{Билет 5. $\lim(x_n y_n)$}

\begin{theorem}
Если $\lim x_n = a$, $\lim y_n = b$, то $\lim (x_n y_n) = a b$.
\end{theorem}

\begin{proofnote}
Используем равенство:
\[
x_n y_n - a b = x_n y_n - x_n b + x_n b - a b = x_n (y_n - b) + (x_n - a) b.
\]

Так как $x_n \to a$, последовательность $\{x_n\}$ ограничена: $\exists M > 0 \; \forall n : |x_n| \leq M$. Возьмём $\varepsilon > 0$. Выберем $N_1$ такой, что $|x_n - a| < \frac{\varepsilon}{2(|b|+1)}$ для всех $n > N_1$. Выберем $N_2$ такой, что $|y_n - b| < \frac{\varepsilon}{2M}$ для всех $n > N_2$. Пусть $N = \max(N_1, N_2)$. Тогда для $n > N$:
\[
|x_n y_n - a b| \leq |x_n| |y_n - b| + |x_n - a| |b| < M \cdot \frac{\varepsilon}{2M} + \frac{\varepsilon}{2(|b|+1)} \cdot |b| \leq \frac{\varepsilon}{2} + \frac{\varepsilon}{2} = \varepsilon.
\]
\end{proofnote}

\section*{Билет 6. $\lim 1/x_n$, $\lim y_n/x_n$}

\begin{theorem}
\begin{enumerate}
    \item Если $\lim x_n = a \neq 0$ и $x_n \neq 0$ для всех $n$, то $\lim \frac{1}{x_n} = \frac{1}{a}$.
    \item Если $\lim y_n = b$, $\lim x_n = a \neq 0$ и $x_n \neq 0$, то $\lim \frac{y_n}{x_n} = \frac{b}{a}$.
\end{enumerate}
\end{theorem}

\begin{proofnote}
\begin{enumerate}
    \item Пусть $a \neq 0$. Покажем, что $\left| \frac{1}{x_n} - \frac{1}{a} \right| \to 0$. Имеем:
    \[
    \left| \frac{1}{x_n} - \frac{1}{a} \right| = \frac{|x_n - a|}{|x_n| |a|}.
    \]
    Так как $x_n \to a$, найдётся $N_1$ такой, что $|x_n - a| < \frac{|a|}{2}$ для всех $n > N_1$. Тогда $|x_n| > \frac{|a|}{2}$. Далее, для $\varepsilon > 0$ выберем $N_2$ такой, что $|x_n - a| < \frac{\varepsilon |a|^2}{2}$ для всех $n > N_2$. Пусть $N = \max(N_1, N_2)$. Тогда для $n > N$:
    \[
    \left| \frac{1}{x_n} - \frac{1}{a} \right| < \frac{\frac{\varepsilon |a|^2}{2}}{\frac{|a|}{2} \cdot |a|} = \varepsilon.
    \]
    \item По пункту 1 $\frac{1}{x_n} \to \frac{1}{a}$, по теореме о пределе произведения $y_n \cdot \frac{1}{x_n} \to b \cdot \frac{1}{a} = \frac{b}{a}$.
\end{enumerate}
\end{proofnote}

\section*{Билет 7. Предельные неравенства}

\begin{theorem}
\begin{enumerate}
    \item Если $x_n \leq y_n$ для всех $n$ и существуют пределы $\lim x_n = a$, $\lim y_n = b$, то $a \leq b$.
    \item Если $x_n \leq y_n \leq z_n$ для всех $n$ и $\lim x_n = \lim z_n = a$, то $\lim y_n = a$.
\end{enumerate}
\end{theorem}

\begin{proofnote}
\begin{enumerate}
    \item Предположим, что $a > b$. Пусть $\varepsilon = \frac{a-b}{2} > 0$. Тогда:
    \[
    \exists N_1 \; \forall n > N_1 : x_n > a - \varepsilon, \quad \exists N_2 \; \forall n > N_2 : y_n < b + \varepsilon.
    \]
    Для $n > \max(N_1, N_2)$ имеем:
    \[
    y_n < b + \varepsilon = \frac{a+b}{2} = a - \varepsilon < x_n,
    \]
    что противоречит условию $x_n \leq y_n$. Значит, $a \leq b$.
    \item Для $\varepsilon > 0$ выберем $N$ такой, что для всех $n > N$ одновременно:
    \[
    a - \varepsilon < x_n \leq y_n \leq z_n < a + \varepsilon.
    \]
    Тогда $|y_n - a| < \varepsilon$, значит, $y_n \to a$.
\end{enumerate}
\end{proofnote}

\section*{Билет 8. Предел монотонной возрастающей последовательности}

\begin{theorem}
Если последовательность $\{x_n\}$ возрастает и ограничена сверху, то она сходится, причём:
\[
\lim_{n \to \infty} x_n = \sup \{x_n\}.
\]
\end{theorem}

\begin{proofnote}
Пусть $E = \{x_n\}$. Так как $E$ ограничено сверху, по теореме о существовании $\sup E$ существует $s = \sup E$. Покажем, что $x_n \to s$. Возьмём $\varepsilon > 0$. По определению супремума существует $N$ такой, что $x_N > s - \varepsilon$. В силу возрастания для всех $n > N$ имеем $x_n \geq x_N > s - \varepsilon$. Кроме того, $x_n \leq s$ для всех $n$. Следовательно, для всех $n > N$:
\[
s - \varepsilon < x_n \leq s < s + \varepsilon \quad \Rightarrow \quad |x_n - s| < \varepsilon.
\]
\end{proofnote}

\section*{Билет 9. Предел монотонной убывающей последовательности}

\begin{theorem}
Если последовательность $\{y_n\}$ убывает и ограничена снизу, то она сходится, причём:
\[
\lim_{n \to \infty} y_n = \inf \{y_n\}.
\]
\end{theorem}

\begin{proofnote}
Аналогично предыдущему. Пусть $i = \inf \{y_n\}$. Возьмём $\varepsilon > 0$. Существует $N$ такой, что $y_N < i + \varepsilon$. В силу убывания для всех $n > N$ имеем $y_n \leq y_N < i + \varepsilon$. Кроме того, $y_n \geq i$. Следовательно, для всех $n > N$:
\[
i \leq y_n < i + \varepsilon \quad \Rightarrow \quad |y_n - i| < \varepsilon.
\]
\end{proofnote}

\section*{Билет 10. Теорема Кантора о вложенных промежутках}

\begin{theorem}
Пусть $\{[a_n, b_n]\}_{n=1}^\infty$ — последовательность вложенных отрезков, т.е.:
\[
a_n \leq a_{n+1} \leq b_{n+1} \leq b_n \quad \forall n.
\]
Пусть также выполняется условие:
\[
b_n - a_n \to 0 \quad \text{при} \quad n \to \infty.
\]
Тогда существует единственная точка $c$, принадлежащая всем отрезкам:
\[
\bigcap_{n=1}^\infty [a_n, b_n] = \{c\}.
\]
\end{theorem}

Рассмотрим два подхода к доказательству.

\textbf{Доказательство через супремум и инфимум (кратко)} 
Множество $\{a_n\}$ ограничено сверху (например, числом $b_1$), а множество $\{b_n\}$ ограничено снизу (числом $a_1$). Пусть:
\[
c = \sup \{a_n\}, \quad d = \inf \{b_n\}.
\]
Так как $a_n \leq b_m$ для любых $n, m$ (в частности, $a_n \leq b_n$), то $c \leq d$. 
Из условия $b_n - a_n \to 0$ следует, что $c = d$.
Любое число $c \leq x \leq d$ принадлежит всем отрезкам. 
В частности, $c$ принадлежит всем отрезкам, так как $a_n \leq c \leq b_n$ для всех $n$.

\textbf{Доказательство через пределы (подробно)} 

1) Существование:

Из вложенности отрезков следует, что последовательность $\{a_n\}$ монотонно возрастает и ограничена сверху (например, $b_1$), 
а последовательность $\{b_n\}$ монотонно убывает и ограничена снизу (например, $a_1$). 
Следовательно, существуют пределы:
\[
a = \lim_{n \to \infty} a_n, \quad b = \lim_{n \to \infty} b_n.
\]

Переходя к пределу в неравенстве $a_n \leq b_n$, получаем:
\[
a \leq b.
\]

Рассмотрим разность $b - a$. 
Из неравенств $a_n \leq a$ и $b \leq b_n$ следует:
\[
b - a \leq b_n - a_n \quad \forall n.
\]

Переходя к пределу при $n \to \infty$, получаем:
\[
0 \leq b - a \leq \lim_{n \to \infty} (b_n - a_n) = 0.
\]

Следовательно, $b - a = 0$, то есть $a = b$. Обозначим эту общую точку через $c = a = b$.

Докажем, что $c \in [a_n, b_n]$ для всех $n$. 
Из монотонности и пределов:
\[
a_n \leq \lim_{n \to \infty} a_n = c = \lim_{n \to \infty} b_n \leq b_n.
\]
Таким образом, $c$ принадлежит всем отрезкам.

2) Единственность:

Предположим, существует ещё одна точка $c_0 \in [a_n, b_n]$ для всех $n$. 
Тогда для любого $n$ выполняется:
\[
|c_0 - c| \leq b_n - a_n.
\]

Переходя к пределу при $n \to \infty$, получаем:
\[
|c_0 - c| \leq \lim_{n \to \infty} (b_n - a_n) = 0.
\]

Следовательно, $c_0 = c$, что доказывает единственность точки пересечения.
\end{proof}

\begin{remark}
Теорема Кантора о вложенных отрезках является одной из эквивалентных формулировок полноты множества действительных чисел.
\end{remark}

\section*{Билет 11. Критерий Коши (необходимость)}

\begin{definition}
Последовательность $\{x_n\}$ называется \textbf{фундаментальной} (сходящейся в себе), если:
\[
\forall \varepsilon > 0 \; \exists N \; \forall n, m > N : |x_n - x_m| < \varepsilon.
\]
\end{definition}

\begin{theorem}[необходимость]
Если последовательность $\{x_n\}$ сходится, то она фундаментальна.
\end{theorem}

\begin{proofnote}
Пусть $x_n \to a$. Возьмём $\varepsilon > 0$. Существует $N$ такой, что для всех $n > N$ выполнено $|x_n - a| < \frac{\varepsilon}{2}$. Тогда для любых $n, m > N$:
\[
|x_n - x_m| \leq |x_n - a| + |a - x_m| < \frac{\varepsilon}{2} + \frac{\varepsilon}{2} = \varepsilon.
\]
\end{proofnote}

\section*{Билет 12. Критерий Коши (достаточность)}

\begin{theorem}
В $\mathbb{R}^m$ всякая фундаментальная последовательность сходится.
\end{theorem}

\begin{proofnote}
\begin{enumerate}
    \item Фундаментальная последовательность ограничена (лемма 6, § 3).
    \item По принципу Больцано–Вейерштрасса из неё можно выделить сходящуюся подпоследовательность $x_{n_k} \to a$.
    \item Покажем, что вся последовательность $x_n \to a$. Возьмём $\varepsilon > 0$. Так как $\{x_n\}$ фундаментальна, существует $N_1$ такой, что $|x_n - x_m| < \frac{\varepsilon}{2}$ для всех $n, m > N_1$. Так как $x_{n_k} \to a$, существует $K$ такой, что $|x_{n_k} - a| < \frac{\varepsilon}{2}$ для всех $k > K$. Выберем $k$ так, чтобы $n_k > \max(N_1, K)$. Тогда для любого $n > N_1$:
    \[
    |x_n - a| \leq |x_n - x_{n_k}| + |x_{n_k} - a| < \frac{\varepsilon}{2} + \frac{\varepsilon}{2} = \varepsilon.
    \]
\end{enumerate}
\end{proofnote}

\section*{Билет 13. Бесконечные пределы, бесконечно большие и малые последовательности}

\begin{definition}
\begin{itemize}
    \item $\lim x_n = +\infty$, если $\forall E > 0 \; \exists N \; \forall n > N : x_n > E$.
    \item $\lim x_n = -\infty$, если $\forall E > 0 \; \exists N \; \forall n > N : x_n < -E$.
    \item $\lim x_n = \infty$, если $\forall E > 0 \; \exists N \; \forall n > N : |x_n| > E$.
    \item Последовательность $\{x_n\}$ называется \textbf{бесконечно малой}, если $\lim x_n = 0$.
    \item Последовательность $\{x_n\}$ называется \textbf{бесконечно большой}, если $\lim x_n = \infty$ (или $\pm\infty$).
\end{itemize}
\end{definition}

\textbf{Связь.} Если $x_n \neq 0$, то $\{x_n\}$ бесконечно большая $\iff$ $\left\{ \frac{1}{x_n} \right\}$ бесконечно малая.

\section*{Билет 14. Число e, последовательность $y_n = \left(1 + \frac{1}{n}\right)^{n+1}$. Конспект \href{https://tagirgaraev.notion.site/20-e-586470cd451f441ba8263cebbc537cda}{ТП}}

\begin{theorem}
Последовательность $y_n = \left(1 + \frac{1}{n}\right)^{n+1}$ убывает и ограничена снизу.
\end{theorem}

\begin{proofnote}
\begin{enumerate}
    \item Ограниченность: $y_n > 1$.
    \item Убывание: Рассмотрим отношение:
    \[
    \frac{y_{n-1}}{y_n} = \frac{\left(1+\frac{1}{n-1}\right)^n}{\left(1+\frac{1}{n}\right)^{n+1}} = \frac{\left(\frac{n}{n-1}\right)^n}{\left(\frac{n+1}{n}\right)^{n+1}}.
    \]
    Используя неравенство Бернулли $(1+x)^r \geq 1+rx$ при $x > -1$, получаем:
    \[
    \left(1 + \frac{1}{n^2-1}\right)^{n+1} \geq 1 + \frac{n+1}{n^2-1}.
    \]
    После преобразований получаем $\frac{y_{n-1}}{y_n} \geq 1$, т.е. $y_{n-1} \geq y_n$.
\end{enumerate}

Следовательно, $y_n$ сходится, и $\lim y_n = \lim \left(1+\frac{1}{n}\right)^n \cdot \lim \left(1+\frac{1}{n}\right) = e \cdot 1 = e$.
\end{proofnote}

\section*{Билет 15. Число e, последовательность $x_n = \left(1 + \frac{1}{n}\right)^n$.
Конспект \href{https://tagirgaraev.notion.site/20-e-586470cd451f441ba8263cebbc537cda}{ТП}}

\begin{theorem}
Последовательность $x_n = \left(1 + \frac{1}{n}\right)^n$ возрастает и ограничена сверху, её предел обозначается через $e$.
\end{theorem}

\begin{proofnote}
Из предыдущего билета $y_n = x_n \cdot \left(1+\frac{1}{n}\right)$ убывает. Так как $x_n = \frac{y_n}{1+\frac{1}{n}}$, а $1+\frac{1}{n} \to 1$, то $x_n$ сходится к тому же пределу $e$. 

Можно также доказать возрастание $x_n$ непосредственно через неравенство Бернулли.
\end{proofnote}

\begin{definition}
\[
e = \lim_{n \to \infty} \left(1 + \frac{1}{n}\right)^n.
\]
\end{definition}

\section*{Билет 16. Предел в терминах окрестностей; подпоследовательности.
Конспект \href{https://tagirgaraev.notion.site/22-711f4c363c1a42bbace4ac75550b7c6c}{ТП}}

\begin{definition}[предела в терминах окрестностей]
$\lim x_n = a$, если для любой окрестности $V_a$ точки $a$ существует $N$ такой, что $x_n \in V_a$ для всех $n > N$.
\end{definition}

\begin{definition}[подпоследовательности]
Пусть $\{n_k\}$ — строго возрастающая последовательность натуральных чисел. Тогда $\{x_{n_k}\}$ называется \textbf{подпоследовательностью} последовательности $\{x_n\}$.
\end{definition}

\begin{theorem}
Если $x_n \to a$, то любая подпоследовательность $x_{n_k} \to a$.
\end{theorem}

\begin{proofnote}
Пусть $V_a$ — окрестность $a$. Существует $N$ такой, что $x_n \in V_a$ для всех $n > N$. Так как $n_k \to \infty$, найдётся $K$ такой, что $n_k > N$ для всех $k > K$. Тогда $x_{n_k} \in V_a$ для всех $k > K$.
\end{proofnote}

\section*{Билет 17. Принцип выбора Больцано–Вейерштрасса.
Конспект \href{https://tagirgaraev.notion.site/22-711f4c363c1a42bbace4ac75550b7c6c}{ТП}}

\begin{theorem}
Из всякой ограниченной последовательности в $\mathbb{R}^m$ можно выделить сходящуюся подпоследовательность.
\end{theorem}

\begin{proofnote}
Пусть последовательность $\{x^{(n)}\}$ ограничена. Тогда все её точки содержатся в некотором замкнутом кубе $I$. Куб $I$ компактен в $\mathbb{R}^m$ (по теореме Гейне–Бореля). В компакте из любой последовательности можно выделить сходящуюся подпоследовательность (свойство секвенциальной компактности).
\end{proofnote}

\section*{Билет 18. Точка сгущения; последовательность, стремящаяся к точке сгущения.
Конспект \href{https://tagirgaraev.notion.site/26-b09738767b8a4f53bd4765cb4045ffcb}{ТП 1} \href{https://tagirgaraev.notion.site/27-a687e7c0c8e6484783c165af5c6898db}{ТП 2}}

\begin{definition}
Точка $a$ называется \textbf{точкой сгущения} (предельной точкой) множества $D$, если в любой проколотой окрестности $\dot{V}_a$ содержится хотя бы одна точка из $D$.
\end{definition}

\begin{theorem}
Точка $a$ является точкой сгущения множества $D$ тогда и только тогда, когда существует последовательность $\{x_n\} \subset D$, $x_n \neq a$, такая, что $x_n \to a$.
\end{theorem}

\begin{proofnote}
\begin{enumerate}
    \item $\Rightarrow$: Для каждого $n$ возьмём проколотую окрестность $\dot{V}_a\left(\frac{1}{n}\right)$. В ней найдётся $x_n \in D$, $x_n \neq a$. По построению $x_n \to a$.
    \item $\Leftarrow$: Если $x_n \to a$, то в любой проколотой окрестности $\dot{V}_a$ лежат почти все члены последовательности, т.е. точки из $D$.
\end{enumerate}
\end{proofnote}

\section*{Билет 19. Предел функции в терминах окрестностей и в терминах $\varepsilon$–$\delta$. Конспект \href{https://tagirgaraev.notion.site/28-d3b8c1558ab8484aae44e39a3c0ec834}{ТП}}

\begin{definition}[по Коши]
Пусть $f: D \subset \mathbb{R} \to \mathbb{R}$, $a$ — предельная точка $D$. Число $A$ называется \textbf{пределом функции} $f$ при $x \to a$, если:
\[
\forall \varepsilon > 0 \; \exists \delta > 0 \; \forall x \in D : 0 < |x - a| < \delta \Rightarrow |f(x) - A| < \varepsilon.
\]
Обозначение: $\lim\limits_{x \to a} f(x) = A$.
\end{definition}

\begin{definition}[по Гейне]
$\lim\limits_{x \to a} f(x) = A$, если для любой последовательности $\{x_n\} \subset D$, $x_n \neq a$, $x_n \to a$, выполняется $f(x_n) \to A$.
\end{definition}

\textbf{Эквивалентность} определений доказывается стандартно.

\section*{Билет 20. Теорема о пределе функции и пределах последовательностей}

\begin{theorem}
Пусть $a$ — предельная точка $D$. Тогда следующие утверждения равносильны:
\begin{enumerate}
    \item $\lim\limits_{x \to a} f(x) = A$ (по Коши).
    \item Для любой последовательность $\{x_n\} \subset D$, $x_n \neq a$, $x_n \to a$, выполнено $\lim f(x_n) = A$.
\end{enumerate}
\end{theorem}

\begin{proofnote}
\begin{enumerate}
    \item $1 \to 2$: Пусть выполнено определение по Коши. Возьмём $\{x_n\} \to a$, $x_n \neq a$. Для $\varepsilon > 0$ найдём $\delta > 0$ из определения Коши. Для этого $\delta$ найдётся $N$ такой, что для всех $n > N$ выполнено $0 < |x_n - a| < \delta$. Тогда $|f(x_n) - A| < \varepsilon$. Значит, $f(x_n) \to A$.
    \item $2 \to 1$: Допустим, что определение по Коши не выполнено. Тогда:
    \[
    \exists \varepsilon > 0 \; \forall \delta > 0 \; \exists x \in D : 0 < |x - a| < \delta \text{ и } |f(x) - A| \geq \varepsilon.
    \]
    Возьмём $\delta_n = \frac{1}{n}$ и выберем $x_n$ соответственно. Тогда $x_n \to a$, $x_n \neq a$, но $|f(x_n) - A| \geq \varepsilon$, противоречие.
\end{enumerate}
\end{proofnote}

% Продолжение с билета 21
\section*{Билет 21. Единственность предела функций; $\lim c$; $\lim cf$; $\lim(f + g)$}

\begin{definition}[Предел функции по Коши]
Пусть $f: D \subset \R \to \R$, $a$ — предельная точка $D$, $A \in \R$. Говорят, что
\[
\lim_{x \to a} f(x) = A,
\]
если
\[
\forall \eps > 0 \; \exists \delta > 0 \; \forall x \in D: \; 0 < |x - a| < \delta \Rightarrow |f(x) - A| < \eps.
\]
\end{definition}

\begin{theorem}[Единственность предела функции]
Если $\lim\limits_{x \to a} f(x) = A$ и $\lim\limits_{x \to a} f(x) = B$, то $A = B$.
\end{theorem}

\begin{proofnote}
Предположим противное: $A \neq B$. Пусть $\eps = \frac{|A - B|}{2} > 0$.
По определению предела:
\begin{itemize}
    \item $\exists \delta_1 > 0: \; 0 < |x - a| < \delta_1 \Rightarrow |f(x) - A| < \eps$,
    \item $\exists \delta_2 > 0: \; 0 < |x - a| < \delta_2 \Rightarrow |f(x) - B| < \eps$.
\end{itemize}
Возьмём $\delta = \min(\delta_1, \delta_2)$. Для любого $x$, удовлетворяющего $0 < |x - a| < \delta$, имеем:
\[
|A - B| \leq |A - f(x)| + |f(x) - B| < \eps + \eps = |A - B|,
\]
что противоречит выбору $\eps$. Следовательно, $A = B$.
\end{proofnote}

\begin{theorem}[Предел постоянной функции]
Пусть $f(x) = c$ для всех $x \in D$. Тогда
\[
\lim_{x \to a} f(x) = c.
\]
\end{theorem}

\begin{proofnote}
Для любого $\eps > 0$ возьмём любое $\delta > 0$. Тогда при $0 < |x - a| < \delta$:
\[
|f(x) - c| = |c - c| = 0 < \eps.
\]
\end{proofnote}

\begin{theorem}[Предел произведения функции на константу]
Если $\lim\limits_{x \to a} f(x) = A$, то для любой константы $c \in \R$:
\[
\lim_{x \to a} (c f(x)) = c A.
\]
\end{theorem}

\begin{proofnote}
Если $c = 0$, утверждение очевидно. Пусть $c \neq 0$.
Для $\eps > 0$ найдём $\delta > 0$ такое, что при $0 < |x - a| < \delta$:
\[
|f(x) - A| < \frac{\eps}{|c|}.
\]
Тогда:
\[
|c f(x) - c A| = |c| \cdot |f(x) - A| < |c| \cdot \frac{\eps}{|c|} = \eps.
\]
\end{proofnote}

\begin{theorem}[Предел суммы функций]
Если $\lim\limits_{x \to a} f(x) = A$ и $\lim\limits_{x \to a} g(x) = B$, то:
\[
\lim_{x \to a} (f(x) + g(x)) = A + B.
\]
\end{theorem}

\begin{proofnote}
Для $\eps > 0$ найдём $\delta_1, \delta_2 > 0$:
\begin{itemize}
    \item $0 < |x - a| < \delta_1 \Rightarrow |f(x) - A| < \frac{\eps}{2}$,
    \item $0 < |x - a| < \delta_2 \Rightarrow |g(x) - B| < \frac{\eps}{2}$.
\end{itemize}
Пусть $\delta = \min(\delta_1, \delta_2)$. Тогда при $0 < |x - a| < \delta$:
\[
|(f(x) + g(x)) - (A + B)| \leq |f(x) - A| + |g(x) - B| < \frac{\eps}{2} + \frac{\eps}{2} = \eps.
\]
\end{proofnote}

\section*{Билет 22. $\lim fg$; $\lim 1/f$; $\lim g/f$}

\begin{theorem}[Предел произведения функций]
Если $\lim\limits_{x \to a} f(x) = A$ и $\lim\limits_{x \to a} g(x) = B$, то:
\[
\lim_{x \to a} (f(x) g(x)) = A B.
\]
\end{theorem}

\begin{proofnote}
Запишем:
\[
f(x) g(x) - A B = f(x) (g(x) - B) + B (f(x) - A).
\]
Поскольку $f(x)$ имеет предел, она ограничена в некоторой проколотой окрестности: $|f(x)| \leq M$.
Для $\eps > 0$ найдём $\delta_1, \delta_2 > 0$:
\begin{itemize}
    \item $0 < |x - a| < \delta_1 \Rightarrow |f(x) - A| < \frac{\eps}{2(|B| + 1)}$,
    \item $0 < |x - a| < \delta_2 \Rightarrow |g(x) - B| < \frac{\eps}{2M}$.
\end{itemize}
Пусть $\delta = \min(\delta_1, \delta_2)$. Тогда:
\[
|f(x) g(x) - A B| \leq |f(x)| \cdot |g(x) - B| + |B| \cdot |f(x) - A| < M \cdot \frac{\eps}{2M} + |B| \cdot \frac{\eps}{2(|B| + 1)} < \eps.
\]
\end{proofnote}

\begin{theorem}[Предел обратной функции]
Если $\lim\limits_{x \to a} f(x) = A \neq 0$, то:
\[
\lim_{x \to a} \frac{1}{f(x)} = \frac{1}{A}.
\]
\end{theorem}

\begin{proofnote}
Пусть $\eps > 0$. Выберем $\delta_1 > 0$ так, чтобы $|f(x) - A| < \frac{|A|}{2}$ при $0 < |x - a| < \delta_1$. Тогда:
\[
|f(x)| \geq |A| - |A - f(x)| > \frac{|A|}{2}.
\]
Теперь для $\eps > 0$ найдём $\delta_2 > 0$:
\[
0 < |x - a| < \delta_2 \Rightarrow |f(x) - A| < \frac{|A|^2 \eps}{2}.
\]
Пусть $\delta = \min(\delta_1, \delta_2)$. Тогда:
\[
\left| \frac{1}{f(x)} - \frac{1}{A} \right| = \frac{|f(x) - A|}{|f(x)| |A|} < \frac{|f(x) - A|}{(|A|/2) |A|} = \frac{2 |f(x) - A|}{|A|^2} < \eps.
\]
\end{proofnote}

\begin{theorem}[Предел частного функций]
Если $\lim\limits_{x \to a} f(x) = A$, $\lim\limits_{x \to a} g(x) = B \neq 0$, то:
\[
\lim_{x \to a} \frac{f(x)}{g(x)} = \frac{A}{B}.
\]
\end{theorem}

\begin{proofnote}
Применяем теоремы о пределе произведения и обратной функции:
\[
\lim_{x \to a} \frac{f(x)}{g(x)} = \lim_{x \to a} \left( f(x) \cdot \frac{1}{g(x)} \right) = A \cdot \frac{1}{B} = \frac{A}{B}.
\]
\end{proofnote}

\section*{Билет 23. $\lim f$, $\lim g$ при $f(x) \leq g(x)$; $\lim f$, $\lim g$, $\lim h$ при $f(x) \leq g(x) \leq h(x)$}

\begin{theorem}[Предельный переход в неравенстве]
Пусть $f(x) \leq g(x)$ в некоторой проколотой окрестности точки $a$, и существуют конечные пределы:
\[
\lim_{x \to a} f(x) = A, \quad \lim_{x \to a} g(x) = B.
\]
Тогда $A \leq B$.
\end{theorem}

\begin{proofnote}
Предположим противное: $A > B$. Возьмём $\eps = \frac{A - B}{2} > 0$.
Найдём $\delta > 0$ такое, что при $0 < |x - a| < \delta$:
\[
|f(x) - A| < \eps, \quad |g(x) - B| < \eps.
\]
Тогда:
\[
f(x) > A - \eps = \frac{A + B}{2}, \quad g(x) < B + \eps = \frac{A + B}{2},
\]
откуда $f(x) > g(x)$, что противоречит условию.
\end{proofnote}

\begin{theorem}[Теорема о сжатой функции]
Пусть в некоторой проколотой окрестности точки $a$:
\[
f(x) \leq g(x) \leq h(x),
\]
и
\[
\lim_{x \to a} f(x) = \lim_{x \to a} h(x) = A.
\]
Тогда $\lim\limits_{x \to a} g(x) = A$.
\end{theorem}

\begin{proofnote}
Для $\eps > 0$ найдём $\delta > 0$ такое, что при $0 < |x - a| < \delta$:
\[
|f(x) - A| < \eps, \quad |h(x) - A| < \eps.
\]
Тогда:
\[
A - \eps < f(x) \leq g(x) \leq h(x) < A + \eps,
\]
следовательно, $|g(x) - A| < \eps$.
\end{proofnote}

\section*{Билет 24. Предел монотонной функции}

\begin{definition}
Функция $f: \langle a, b \rangle \to \R$ называется:
\begin{itemize}
    \item возрастающей, если $x_1 < x_2 \Rightarrow f(x_1) \leq f(x_2)$,
    \item строго возрастающей, если $x_1 < x_2 \Rightarrow f(x_1) < f(x_2)$,
    \item убывающей, если $x_1 < x_2 \Rightarrow f(x_1) \geq f(x_2)$,
    \item строго убывающей, если $x_1 < x_2 \Rightarrow f(x_1) > f(x_2)$.
\end{itemize}
\end{definition}

\begin{theorem}[О пределе монотонной функции]
Пусть $f: \langle a, b \rangle \to \R$ возрастает и ограничена сверху. Тогда существует конечный предел:
\[
\lim_{x \to b^-} f(x) = \sup_{x \in \langle a, b \rangle} f(x).
\]
Аналогично, если $f$ убывает и ограничена снизу, то:
\[
\lim_{x \to b^-} f(x) = \inf_{x \in \langle a, b \rangle} f(x).
\]
\end{theorem}

\begin{proofnote}
Пусть $M = \sup\limits_{x \in \langle a, b \rangle} f(x)$. Для любого $\eps > 0$ найдётся $x_0 \in \langle a, b \rangle$ такое, что:
\[
f(x_0) > M - \eps.
\]
Тогда для всех $x \in (x_0, b)$ в силу возрастания:
\[
M - \eps < f(x_0) \leq f(x) \leq M.
\]
Положим $\delta = b - x_0$. Тогда при $0 < b - x < \delta$ (т.е. $x > x_0$) имеем:
\[
|f(x) - M| < \eps.
\]
Следовательно, $\lim\limits_{x \to b^-} f(x) = M$.
\end{proofnote}

\section*{Билет 25. Критерий Коши существования конечного предела функции (необходимость)}

\begin{theorem}[Критерий Коши для функции]
Пусть $f: D \subset \R \to \R$, $a$ — предельная точка $D$. Конечный предел $\lim\limits_{x \to a} f(x)$ существует тогда и только тогда, когда:
\[
\forall \eps > 0 \; \exists \delta > 0 \; \forall x_1, x_2 \in D: \; 0 < |x_1 - a| < \delta, \; 0 < |x_2 - a| < \delta \Rightarrow |f(x_1) - f(x_2)| < \eps.
\]
\end{theorem}

\begin{theorem}[Необходимость условия Коши]
Если существует конечный предел $\lim\limits_{x \to a} f(x) = A$, то выполняется условие Коши.
\end{theorem}

\begin{proofnote}
Для $\eps > 0$ найдём $\delta > 0$ такое, что при $0 < |x - a| < \delta$:
\[
|f(x) - A| < \frac{\eps}{2}.
\]
Тогда для любых $x_1, x_2$, удовлетворяющих $0 < |x_1 - a| < \delta$, $0 < |x_2 - a| < \delta$:
\[
|f(x_1) - f(x_2)| \leq |f(x_1) - A| + |A - f(x_2)| < \frac{\eps}{2} + \frac{\eps}{2} = \eps.
\]
\end{proofnote}

\section*{Билет 26. Критерий Коши существования конечного предела функции (достаточность)}

\begin{theorem}[Достаточность условия Коши]
Если для функции $f$ выполняется условие Коши, то существует конечный предел $\lim\limits_{x \to a} f(x)$.
\end{theorem}

\begin{proofnote}
\begin{enumerate}
    \item Покажем, что $f$ ограничена в некоторой проколотой окрестности $a$.
    Возьмём $\eps = 1$. Найдём $\delta_0 > 0$ такое, что для любых $x_1, x_2$ из проколотой $\delta_0$-окрестности:
    \[
    |f(x_1) - f(x_2)| < 1.
    \]
    Зафиксируем $x_0$ из этой окрестности. Тогда для любого $x$:
    \[
    |f(x)| \leq |f(x) - f(x_0)| + |f(x_0)| < 1 + |f(x_0)|.
    \]
    
    \item Рассмотрим произвольную последовательность $\{x_n\} \subset D \setminus \{a\}$, $x_n \to a$.
    Покажем, что $\{f(x_n)\}$ сходится.
    Для $\eps > 0$ найдём $\delta > 0$ из условия Коши. Так как $x_n \to a$, найдётся $N$ такое, что при $n > N$:
    \[
    0 < |x_n - a| < \delta.
    \]
    Тогда для $n, m > N$:
    \[
    |f(x_n) - f(x_m)| < \eps.
    \]
    Значит, $\{f(x_n)\}$ — фундаментальная, следовательно, сходится.
    
    \item Покажем, что все такие последовательности сходятся к одному пределу.
    Пусть $x_n \to a$, $y_n \to a$, $f(x_n) \to A$, $f(y_n) \to B$. Рассмотрим смешанную последовательность $z_{2n-1} = x_n$, $z_{2n} = y_n$. Тогда $z_n \to a$, и $\{f(z_n)\}$ сходится (по п.2). Отсюда $A = B$.
    
    \item По определению предела по Гейне, $\lim\limits_{x \to a} f(x) = A$.
\end{enumerate}
\end{proofnote}

\section*{Билет 27. Шкала бесконечно малых}

\begin{definition}
Функция $\alpha(x)$ называется бесконечно малой при $x \to a$, если:
\[
\lim_{x \to a} \alpha(x) = 0.
\]
\end{definition}

\begin{definition}
Говорят, что $\alpha(x) = o(\beta(x))$ при $x \to a$, если:
\[
\lim_{x \to a} \frac{\alpha(x)}{\beta(x)} = 0.
\]
\end{definition}

\begin{theorem}[Шкала бесконечно малых при $x \to 0$]
Для натуральных $n, m$:
\begin{itemize}
    \item $x^n = o(x^m)$ при $n > m$,
    \item $\sin x \sim x$, $\tg x \sim x$, $\arcsin x \sim x$, $\arctg x \sim x$,
    \item $1 - \cos x \sim \frac{x^2}{2}$,
    \item $\ln(1 + x) \sim x$,
    \item $e^x - 1 \sim x$,
    \item $(1 + x)^\alpha - 1 \sim \alpha x$.
\end{itemize}
\end{theorem}

\section*{Билет 28. Существование неравенства для $\ln(1+x)$}

\begin{theorem}
Для любого $x > -1$, $x \neq 0$, верно неравенство:
\[
\frac{x}{1+x} < \ln(1+x) < x.
\]
\end{theorem}

\begin{proofnote}
Рассмотрим функцию $f(t) = \ln(1+t)$. По формуле Лагранжа на отрезке $[0, x]$ (при $x > 0$):
\[
\ln(1+x) = \ln(1) + \frac{1}{1+\theta x} \cdot x, \quad \theta \in (0,1).
\]
Так как $1 < 1+\theta x < 1+x$, то:
\[
\frac{x}{1+x} < \ln(1+x) < x.
\]
Для $x \in (-1, 0)$ аналогично на отрезке $[x, 0]$:
\[
\ln(1+x) = \ln(1) + \frac{1}{1+\theta x} \cdot x, \quad \theta \in (0,1),
\]
и так как $1+x < 1+\theta x < 1$, то снова:
\[
\frac{x}{1+x} < \ln(1+x) < x.
\]
\end{proofnote}

\section*{Билет 29. Существование неравенства для $e^x$}

\begin{theorem}
Для любого $x \neq 0$ верно неравенство:
\[
1 + x < e^x \quad \text{при } x > 0, \quad e^x < 1 + x \quad \text{при } x < 0.
\]
\end{theorem}

\begin{proofnote}
Рассмотрим $f(t) = e^t$. По формуле Лагранжа на отрезке $[0, x]$ (при $x > 0$):
\[
e^x = 1 + e^{\theta x} \cdot x, \quad \theta \in (0,1).
\]
Так как $e^{\theta x} > 1$, то $e^x > 1 + x$.
При $x < 0$ на отрезке $[x, 0]$:
\[
1 = e^x + e^{\theta x} \cdot (-x), \quad \theta \in (0,1),
\]
откуда $e^x = 1 - e^{\theta x} \cdot |x| < 1 - |x| = 1 + x$.
\end{proofnote}

\section*{Билет 30. Существование неравенства для $(1+x)^{1/x}$}

\begin{theorem}
Для $x > 0$ верно:
\[
\left(1 + \frac{1}{x}\right)^x < e < \left(1 + \frac{1}{x}\right)^{x+1}.
\]
\end{theorem}

\begin{proofnote}
Из неравенства для логарифма (билет 28):
\[
\frac{1}{x+1} < \ln\left(1 + \frac{1}{x}\right) < \frac{1}{x}.
\]
Умножаем на $x > 0$:
\[
\frac{x}{x+1} < x \ln\left(1 + \frac{1}{x}\right) < 1.
\]
Экспонируем:
\[
e^{\frac{x}{x+1}} < \left(1 + \frac{1}{x}\right)^x < e.
\]
Так как $e^{\frac{x}{x+1}} = e^{1 - \frac{1}{x+1}} < e$, левое неравенство усилить нельзя.
Для правого: умножим исходное неравенство на $x+1 > 0$:
\[
1 < (x+1) \ln\left(1 + \frac{1}{x}\right) < \frac{x+1}{x}.
\]
Экспонируем:
\[
e < \left(1 + \frac{1}{x}\right)^{x+1} < e^{\frac{x+1}{x}}.
\]
\end{proofnote}

\section*{Билет 31. $\lim\limits_{x \to 0} \frac{\ln(1+x)}{x}$}

\begin{theorem}
\[
\lim_{x \to 0} \frac{\ln(1+x)}{x} = 1.
\]
\end{theorem}

\begin{proofnote}
Используем неравенство из билета 28:
\[
\frac{1}{1+x} < \frac{\ln(1+x)}{x} < 1 \quad \text{при } x > 0,
\]
\[
1 < \frac{\ln(1+x)}{x} < \frac{1}{1+x} \quad \text{при } -1 < x < 0.
\]
По теореме о сжатой функции:
\[
\lim_{x \to 0} \frac{\ln(1+x)}{x} = 1.
\]
\end{proofnote}

\section*{Билет 32. $\lim\limits_{x \to 0} \frac{e^x - 1}{x}$}

\begin{theorem}
\[
\lim_{x \to 0} \frac{e^x - 1}{x} = 1.
\]
\end{theorem}

\begin{proofnote}
Сделаем замену $y = e^x - 1$, тогда $x = \ln(1+y)$, $y \to 0$ при $x \to 0$.
\[
\frac{e^x - 1}{x} = \frac{y}{\ln(1+y)} \to \frac{1}{\lim\limits_{y \to 0} \frac{\ln(1+y)}{y}} = 1.
\]
\end{proofnote}

\section*{Билет 33. $\lim\limits_{x \to 0} (1 + x)^{1/x}$}

\begin{theorem}
\[
\lim_{x \to 0} (1 + x)^{1/x} = e.
\]
\end{theorem}

\begin{proofnote}
\[
(1 + x)^{1/x} = e^{\frac{\ln(1+x)}{x}}.
\]
Так как $\frac{\ln(1+x)}{x} \to 1$ (билет 31), по непрерывности экспоненты:
\[
\lim_{x \to 0} e^{\frac{\ln(1+x)}{x}} = e^1 = e.
\]
\end{proofnote}

\section*{Билет 34. $\lim\limits_{x \to 0} (1 + x)^r$}

\begin{theorem}
Для любого $r \in \R$:
\[
\lim_{x \to 0} (1 + x)^r = 1.
\]
\end{theorem}

\begin{proofnote}
\[
(1 + x)^r = e^{r \ln(1+x)}.
\]
Так как $\ln(1+x) \to 0$, то $r \ln(1+x) \to 0$, и по непрерывности экспоненты:
\[
\lim_{x \to 0} e^{r \ln(1+x)} = e^0 = 1.
\]
\end{proofnote}

\section*{Билет 35. $\lim\limits_{x \to 0} \frac{\sin x}{x}$}

\begin{theorem}
\[
\lim_{x \to 0} \frac{\sin x}{x} = 1.
\]
\end{theorem}

\begin{proofnote}
Для $x \in (0, \pi/2)$ имеем (см. геометрическое построение):
\[
\sin x < x < \tg x.
\]
Делим на $\sin x > 0$:
\[
1 < \frac{x}{\sin x} < \frac{1}{\cos x}.
\]
Переворачиваем:
\[
\cos x < \frac{\sin x}{x} < 1.
\]
Так как $\cos x \to 1$ при $x \to 0$, по теореме о сжатой функции:
\[
\lim_{x \to 0} \frac{\sin x}{x} = 1.
\]
Для $x < 0$ замена $y = -x$ даёт тот же результат.
\end{proofnote}

\section*{Билет 36. Непрерывность функции в точке; арифметические операции над непрерывными функциями}

\begin{definition}[Непрерывность в точке по Коши]
Функция $f: D \subset \R \to \R$ непрерывна в точке $a \in D$, если:
\[
\forall \eps > 0 \; \exists \delta > 0 \; \forall x \in D: \; |x - a| < \delta \Rightarrow |f(x) - f(a)| < \eps.
\]
\end{definition}

\begin{theorem}[Сумма непрерывных функций]
Если $f$ и $g$ непрерывны в точке $a$, то $f + g$ также непрерывна в $a$.
\end{theorem}

\begin{proofnote}
Для $\eps > 0$ найдём $\delta_1, \delta_2 > 0$:
\begin{itemize}
    \item $|x - a| < \delta_1 \Rightarrow |f(x) - f(a)| < \frac{\eps}{2}$,
    \item $|x - a| < \delta_2 \Rightarrow |g(x) - g(a)| < \frac{\eps}{2}$.
\end{itemize}
Пусть $\delta = \min(\delta_1, \delta_2)$. Тогда:
\[
|(f+g)(x) - (f+g)(a)| \leq |f(x) - f(a)| + |g(x) - g(a)| < \eps.
\]
\end{proofnote}

\begin{theorem}[Произведение непрерывных функций]
Если $f$ и $g$ непрерывны в точке $a$, то $f \cdot g$ непрерывна в $a$.
\end{theorem}

\begin{proofnote}
Используем представление:
\[
f(x)g(x) - f(a)g(a) = f(x)(g(x) - g(a)) + g(a)(f(x) - f(a)).
\]
Так как $f$ непрерывна, она ограничена в окрестности $a$: $|f(x)| \leq M$.
Для $\eps > 0$ найдём $\delta_1, \delta_2 > 0$:
\begin{itemize}
    \item $|x - a| < \delta_1 \Rightarrow |f(x) - f(a)| < \frac{\eps}{2(|g(a)| + 1)}$,
    \item $|x - a| < \delta_2 \Rightarrow |g(x) - g(a)| < \frac{\eps}{2M}$.
\end{itemize}
Пусть $\delta = \min(\delta_1, \delta_2)$. Тогда:
\[
|f(x)g(x) - f(a)g(a)| < M \cdot \frac{\eps}{2M} + |g(a)| \cdot \frac{\eps}{2(|g(a)| + 1)} < \eps.
\]
\end{proofnote}

\begin{theorem}[Частное непрерывных функций]
Если $f$ и $g$ непрерывны в точке $a$ и $g(a) \neq 0$, то $\frac{f}{g}$ непрерывна в $a$.
\end{theorem}

\begin{proofnote}
Сначала покажем, что $\frac{1}{g}$ непрерывна в $a$.
Для $\eps > 0$ выберем $\delta_1 > 0$ так, чтобы $|g(x) - g(a)| < \frac{|g(a)|}{2}$ при $|x - a| < \delta_1$. Тогда:
\[
|g(x)| \geq |g(a)| - |g(a) - g(x)| > \frac{|g(a)|}{2}.
\]
Теперь для $\eps > 0$ найдём $\delta_2 > 0$:
\[
|x - a| < \delta_2 \Rightarrow |g(x) - g(a)| < \frac{|g(a)|^2 \eps}{2}.
\]
Пусть $\delta = \min(\delta_1, \delta_2)$. Тогда:
\[
\left| \frac{1}{g(x)} - \frac{1}{g(a)} \right| = \frac{|g(x) - g(a)|}{|g(x)| |g(a)|} < \frac{|g(x) - g(a)|}{(|g(a)|/2) |g(a)|} = \frac{2 |g(x) - g(a)|}{|g(a)|^2} < \eps.
\]
Теперь $\frac{f}{g} = f \cdot \frac{1}{g}$ непрерывна как произведение непрерывных.
\end{proofnote}

\section*{Билет 37. Односторонняя непрерывность функции; Классификация разрывов}

\begin{definition}[Односторонняя непрерывность]
Функция $f: D \subset \R \to \R$ непрерывна \textbf{слева} в точке $a \in D$, если:
\[
\lim_{x \to a^-} f(x) = f(a).
\]
Аналогично определяется непрерывность \textbf{справа}.
\end{definition}

\begin{definition}[Точка разрыва]
Точка $a$ называется точкой разрыва функции $f$, если $f$ не является непрерывной в $a$.
\end{definition}

\begin{theorem}[Классификация разрывов]
\begin{enumerate}
    \item \textbf{Устранимый разрыв}: существуют конечные односторонние пределы $f(a^-) = f(a^+)$, но не равны $f(a)$ (или $f(a)$ не определено).
    \item \textbf{Разрыв первого рода (скачок)}: существуют конечные односторонние пределы $f(a^-)$ и $f(a^+)$, но они не равны.
    \item \textbf{Разрыв второго рода}: хотя бы один из односторонних пределов бесконечен или не существует.
\end{enumerate}
\end{theorem}

\section*{Билет 38. Непрерывность и разрывы монотонной функции}

\begin{theorem}
Пусть $f: \langle a, b \rangle \to \R$ монотонна. Тогда:
\begin{enumerate}
    \item $f$ не может иметь разрывов второго рода.
    \item В каждой точке $x_0 \in (a, b)$ существуют конечные пределы:
    \[
    f(x_0^-) = \lim_{x \to x_0^-} f(x), \quad f(x_0^+) = \lim_{x \to x_0^+} f(x),
    \]
    причём $f(x_0^-) \leq f(x_0) \leq f(x_0^+)$ (для возрастающей).
    \item $f$ непрерывна в точке $x_0$ тогда и только тогда, когда $f(x_0^-) = f(x_0) = f(x_0^+)$.
\end{enumerate}
\end{theorem}

\begin{proofnote}
Пусть $f$ возрастает. Для $x_0 \in (a, b)$ множество $f(x)$ при $x < x_0$ ограничено сверху $f(x_0)$. По теореме о пределе монотонной функции (билет 24) существует конечный предел $f(x_0^-) = \sup_{x < x_0} f(x)$. Аналогично, $f(x_0^+) = \inf_{x > x_0} f(x)$.
Так как $f$ возрастает, $f(x_0^-) \leq f(x_0) \leq f(x_0^+)$.
Если $f(x_0^-) = f(x_0) = f(x_0^+)$, то $f$ непрерывна в $x_0$.
Если же $f(x_0^-) < f(x_0)$ или $f(x_0) < f(x_0^+)$, то это разрыв первого рода.
\end{proofnote}

\section*{Билет 39. Непрерывность суперпозиции непрерывных функций}

\begin{theorem}
Если $f: D \subset \R \to \R$ непрерывна в точке $a \in D$, а $g: E \subset \R \to \R$ непрерывна в точке $f(a) \in E$, и $f(D) \subset E$, то композиция $g \circ f$ непрерывна в точке $a$.
\end{theorem}

\begin{proofnote}
Пусть $\eps > 0$. Так как $g$ непрерывна в $f(a)$, найдём $\delta_1 > 0$:
\[
|y - f(a)| < \delta_1 \Rightarrow |g(y) - g(f(a))| < \eps.
\]
Так как $f$ непрерывна в $a$, для $\delta_1 > 0$ найдём $\delta > 0$:
\[
|x - a| < \delta \Rightarrow |f(x) - f(a)| < \delta_1.
\]
Тогда:
\[
|g(f(x)) - g(f(a))| < \eps.
\]
\end{proofnote}

\section*{Билет 40. Непрерывность $\ln x$}

\begin{theorem}
Функция $\ln x$ непрерывна на $(0, +\infty)$.
\end{theorem}

\begin{proofnote}
Пусть $a > 0$. Для любого $\eps > 0$ найдём $\delta > 0$ такое, что при $|x - a| < \delta$:
\[
|\ln x - \ln a| < \eps.
\]
Заметим, что:
\[
|\ln x - \ln a| = \left| \ln \frac{x}{a} \right|.
\]
Потребуем, чтобы:
\[
e^{-\eps} < \frac{x}{a} < e^{\eps}.
\]
Тогда:
\[
a(e^{-\eps} - 1) < x - a < a(e^{\eps} - 1).
\]
Выберем $\delta = a \min(1 - e^{-\eps}, e^{\eps} - 1)$. Тогда при $|x - a| < \delta$:
\[
|\ln x - \ln a| < \eps.
\]
\end{proofnote}

\section*{Билет 41. Непрерывность $e^x$}

\begin{theorem}
Функция $e^x$ непрерывна на $\R$.
\end{theorem}

\begin{proofnote}
Пусть $a \in \R$. Для $\eps > 0$ найдём $\delta > 0$ такое, что при $|x - a| < \delta$:
\[
|e^x - e^a| < \eps.
\]
Заметим, что:
\[
|e^x - e^a| = e^a |e^{x-a} - 1|.
\]
Из непрерывности $e^t$ в нуле: найдём $\delta_1 > 0$:
\[
|t| < \delta_1 \Rightarrow |e^t - 1| < \frac{\eps}{e^a}.
\]
Положим $\delta = \delta_1$. Тогда при $|x - a| < \delta$:
\[
|e^x - e^a| < e^a \cdot \frac{\eps}{e^a} = \eps.
\]
\end{proofnote}

\section*{Билет 42. Непрерывность $x^r$}

\begin{theorem}
Степенная функция $x^r$ непрерывна на своей области определения:
\begin{itemize}
    \item $(0, +\infty)$ при любом $r \in \R$,
    \item $[0, +\infty)$ при $r > 0$,
    \item $\R \setminus \{0\}$ при $r$ целом отрицательном и т.д.
\end{itemize}
\end{theorem}

\begin{proofnote}
Запишем:
\[
x^r = e^{r \ln x}.
\]
Так как $\ln x$ непрерывен на $(0, +\infty)$ (билет 40), а экспонента непрерывна на $\R$ (билет 41), композиция $e^{r \ln x}$ непрерывна на $(0, +\infty)$.
При $r > 0$ докажем непрерывность в нуле справа:
Пусть $x_n \to 0+$. Тогда $\ln x_n \to -\infty$, $r \ln x_n \to -\infty$, $e^{r \ln x_n} \to 0 = 0^r$.
\end{proofnote}

\section*{Билет 43. Непрерывность $\sin x$}

\begin{theorem}
Функция $\sin x$ непрерывна на $\R$.
\end{theorem}

\begin{proofnote}
Используем формулу разности синусов:
\[
|\sin x - \sin a| = \left| 2 \sin \frac{x - a}{2} \cos \frac{x + a}{2} \right| \leq 2 \left| \sin \frac{x - a}{2} \right| \leq |x - a|.
\]
Для $\eps > 0$ возьмём $\delta = \eps$. Тогда при $|x - a| < \delta$:
\[
|\sin x - \sin a| \leq |x - a| < \eps.
\]
\end{proofnote}

\section*{Билет 44. Непрерывность $\cos x$, $\tg x$, $\ctg x$}

\begin{theorem}
\begin{enumerate}
    \item $\cos x$ непрерывен на $\R$.
    \item $\tg x = \frac{\sin x}{\cos x}$ непрерывен на $\R \setminus \left\{ \frac{\pi}{2} + k\pi \right\}$.
    \item $\ctg x = \frac{\cos x}{\sin x}$ непрерывен на $\R \setminus \{k\pi\}$.
\end{enumerate}
\end{theorem}

\begin{proofnote}
\begin{enumerate}
    \item $\cos x = \sin\left(\frac{\pi}{2} - x\right)$ — композиция непрерывных.
    \item $\tg x$ — частное непрерывных функций, знаменатель не обращается в ноль в указанных точках.
    \item Аналогично для $\ctg x$.
\end{enumerate}
\end{proofnote}

\section*{Билет 45. Теорема об отображении отрезка}

\begin{theorem}[Вейерштрасс]
Если функция $f$ непрерывна на отрезке $[a, b]$, то её образ $f([a, b])$ является отрезком.
\end{theorem}

\begin{proofnote}
По теореме Вейерштрасса $f$ ограничена на $[a, b]$.
Пусть $m = \inf\limits_{[a,b]} f(x)$, $M = \sup\limits_{[a,b]} f(x)$.
По теореме Больцано–Коши (билет 47) для любого $y \in [m, M]$ найдётся $c \in [a, b]$ такое, что $f(c) = y$.
Следовательно, $f([a, b]) = [m, M]$.
\end{proofnote}

\section*{Билет 46. Теорема об обращении непрерывной функции в ноль}

\begin{theorem}[Больцано–Коши, первая форма]
Если $f$ непрерывна на $[a, b]$ и $f(a) \cdot f(b) < 0$, то существует точка $c \in (a, b)$ такая, что $f(c) = 0$.
\end{theorem}

\begin{proofnote}
Метод половинного деления.
Положим $a_0 = a, b_0 = b$.
На каждом шаге $n$ рассматриваем середину $c_n = \frac{a_n + b_n}{2}$.
Если $f(c_n) = 0$, доказано.
Иначе выбираем $[a_{n+1}, b_{n+1}]$ так, чтобы $f(a_{n+1}) \cdot f(b_{n+1}) < 0$.
Получаем последовательность вложенных отрезков, длина которых стремится к нулю.
По теореме Кантора существует единственная точка $c$, принадлежащая всем отрезкам.
По непрерывности $f(c) = 0$.
\end{proofnote}

\section*{Билет 47. Теорема о промежуточном значении}

\begin{theorem}[Больцано–Коши]
Если $f$ непрерывна на $[a, b]$, то для любого числа $C$ между $f(a)$ и $f(b)$ найдётся точка $c \in (a, b)$ такая, что $f(c) = C$.
\end{theorem}

\begin{proofnote}
Рассмотрим функцию $g(x) = f(x) - C$.
Тогда $g(a) \cdot g(b) < 0$.
По теореме билета 46 найдётся $c \in (a, b)$: $g(c) = 0$, т.е. $f(c) = C$.
\end{proofnote}

\section*{Билет 48. Существование обратной непрерывной функции}

\begin{theorem}
Пусть $f$ непрерывна и строго монотонна на промежутке $\langle a, b \rangle$.
Тогда:
\begin{enumerate}
    \item $f$ обратима, $f^{-1}$ определена на промежутке $f(\langle a, b \rangle)$.
    \item $f^{-1}$ строго монотонна того же типа.
    \item $f^{-1}$ непрерывна.
\end{enumerate}
\end{theorem}

\begin{proofnote}
\begin{enumerate}
    \item Строгая монотонность обеспечивает обратимость.
    \item Монотонность $f^{-1}$ следует из монотонности $f$.
    \item Непрерывность $f^{-1}$ следует из теоремы о сохранении промежутка и теоремы о разрывах монотонной функции (билет 38).
\end{enumerate}
\end{proofnote}

\section*{Билет 49. Непрерывность $\arcsin x$, $\arccos x$, $\arctg x$, $\arcctg x$}

\begin{theorem}
Обратные тригонометрические функции непрерывны на своих областях определения:
\begin{itemize}
    \item $\arcsin x$ на $[-1, 1]$,
    \item $\arccos x$ на $[-1, 1]$,
    \item $\arctg x$ на $\R$,
    \item $\arcctg x$ на $\R$.
\end{itemize}
\end{theorem}

\begin{proofnote}
Эти функции являются обратными к непрерывным строго монотонным функциям:
\begin{itemize}
    \item $\arcsin$ — обратная к $\sin$ на $[-\pi/2, \pi/2]$,
    \item $\arccos$ — обратная к $\cos$ на $[0, \pi]$,
    \item $\arctg$ — обратная к $\tg$ на $(-\pi/2, \pi/2)$,
    \item $\arcctg$ — обратная к $\ctg$ на $(0, \pi)$.
\end{itemize}
По теореме билета 48 они непрерывны.
\end{proofnote}

\section*{Билет 50. Равномерная непрерывность функций; Теорема Кантора}

\begin{definition}[Равномерная непрерывность]
Функция $f: D \subset \R \to \R$ равномерно непрерывна на $D$, если:
\[
\forall \eps > 0 \; \exists \delta > 0 \; \forall x_1, x_2 \in D: \; |x_1 - x_2| < \delta \Rightarrow |f(x_1) - f(x_2)| < \eps.
\]
\end{definition}

\begin{theorem}[Кантора]
Если функция непрерывна на отрезке $[a, b]$, то она равномерно непрерывна на $[a, b]$.
\end{theorem}

\begin{proofnote}
От противного. Пусть $f$ не равномерно непрерывна.
Тогда $\exists \eps > 0$ такое, что $\forall \delta > 0$ найдутся $x, y \in [a, b]$: $|x - y| < \delta$, но $|f(x) - f(y)| \geq \eps$.
Возьмём $\delta_n = \frac{1}{n}$. Получим последовательности $\{x_n\}, \{y_n\}$ такие, что $|x_n - y_n| < \frac{1}{n}$, но $|f(x_n) - f(y_n)| \geq \eps$.
По теореме Больцано–Вейерштрасса выделим сходящуюся подпоследовательность $\{x_{n_k}\} \to c$.
Тогда и $\{y_{n_k}\} \to c$.
По непрерывности $f$ в точке $c$:
\[
|f(x_{n_k}) - f(y_{n_k})| \to |f(c) - f(c)| = 0,
\]
что противоречит условию $|f(x_{n_k}) - f(y_{n_k})| \geq \eps$.
\end{proofnote}

\section*{Билет 51. Определение производной и односторонней производной}

\begin{definition}[Производная]
Функция $f: \langle a, b \rangle \to \R$ имеет производную в точке $x_0 \in \langle a, b \rangle$, если существует конечный предел:
\[
f'(x_0) = \lim_{x \to x_0} \frac{f(x) - f(x_0)}{x - x_0} = \lim_{h \to 0} \frac{f(x_0 + h) - f(x_0)}{h}.
\]
\end{definition}

\begin{definition}[Односторонняя производная]
Правая производная:
\[
f'_+(x_0) = \lim_{h \to 0^+} \frac{f(x_0 + h) - f(x_0)}{h}.
\]
Левая производная:
\[
f'_-(x_0) = \lim_{h \to 0^-} \frac{f(x_0 + h) - f(x_0)}{h}.
\]
\end{definition}

\section*{Билет 52. Дифференцируемость функции; связь дифференцируемости и существования производной}

\begin{definition}[Дифференцируемость]
Функция $f$ дифференцируема в точке $x_0$, если существует такое число $A$, что:
\[
f(x_0 + h) = f(x_0) + A h + o(h), \quad h \to 0.
\]
\end{definition}

\begin{theorem}
Функция дифференцируема в точке $x_0$ тогда и только тогда, когда существует конечная производная $f'(x_0)$, причём $A = f'(x_0)$.
\end{theorem}

\begin{proofnote}
\begin{enumerate}
    \item Если $f$ дифференцируема, то:
    \[
    \frac{f(x_0 + h) - f(x_0)}{h} = A + \frac{o(h)}{h} \to A.
    \]
    \item Если существует $f'(x_0)$, то:
    \[
    \frac{f(x_0 + h) - f(x_0)}{h} = f'(x_0) + \alpha(h), \quad \alpha(h) \to 0,
    \]
    откуда:
    \[
    f(x_0 + h) = f(x_0) + f'(x_0) h + \alpha(h) h.
    \]
\end{enumerate}
\end{proofnote}

\section*{Билет 53. $(cf)'$, $(f + g)'$, $(fg)'$}

\begin{theorem}
Если $f$ и $g$ дифференцируемы в точке $x$, то:
\begin{enumerate}
    \item $(cf)'(x) = c f'(x)$,
    \item $(f + g)'(x) = f'(x) + g'(x)$,
    \item $(fg)'(x) = f'(x) g(x) + f(x) g'(x)$.
\end{enumerate}
\end{theorem}

\begin{proofnote}
\begin{enumerate}
    \item \[
    \frac{cf(x+h) - cf(x)}{h} = c \frac{f(x+h) - f(x)}{h} \to c f'(x).
    \]
    \item \[
    \frac{(f+g)(x+h) - (f+g)(x)}{h} = \frac{f(x+h) - f(x)}{h} + \frac{g(x+h) - g(x)}{h} \to f'(x) + g'(x).
    \]
    \item \[
    \frac{f(x+h)g(x+h) - f(x)g(x)}{h} = f(x+h) \frac{g(x+h) - g(x)}{h} + g(x) \frac{f(x+h) - f(x)}{h} \to f(x) g'(x) + g(x) f'(x).
    \]
\end{enumerate}
\end{proofnote}

\section*{Билет 54. $(1/f)'$, $(g/f)'$}

\begin{theorem}
Если $f$ дифференцируема в точке $x$ и $f(x) \neq 0$, то:
\begin{enumerate}
    \item $\left( \frac{1}{f} \right)'(x) = -\frac{f'(x)}{f^2(x)}$,
    \item $\left( \frac{g}{f} \right)'(x) = \frac{g'(x) f(x) - g(x) f'(x)}{f^2(x)}$.
\end{enumerate}
\end{theorem}

\begin{proofnote}
\begin{enumerate}
    \item \[
    \frac{\frac{1}{f(x+h)} - \frac{1}{f(x)}}{h} = \frac{f(x) - f(x+h)}{h f(x+h) f(x)} \to -\frac{f'(x)}{f^2(x)}.
    \]
    \item Применяем правило произведения: $\frac{g}{f} = g \cdot \frac{1}{f}$, тогда:
    \[
    \left( \frac{g}{f} \right)' = g' \cdot \frac{1}{f} + g \cdot \left( -\frac{f'}{f^2} \right) = \frac{g' f - g f'}{f^2}.
    \]
\end{enumerate}
\end{proofnote}

\section*{Билет 55. $(g(f))'$, $(f^{-1})'$}

\begin{theorem}[Производная композиции]
Если $f$ дифференцируема в точке $x$, а $g$ дифференцируема в точке $f(x)$, то:
\[
(g \circ f)'(x) = g'(f(x)) \cdot f'(x).
\]
\end{theorem}

\begin{proofnote}
Пусть $y = f(x)$, $k = f(x+h) - f(x)$. Тогда:
\[
\frac{g(f(x+h)) - g(f(x))}{h} = \frac{g(y+k) - g(y)}{k} \cdot \frac{k}{h}.
\]
Если $k \neq 0$, то при $h \to 0$ первый множитель стремится к $g'(y)$, второй — к $f'(x)$.
Если $k = 0$, то левая часть равна нулю, и формула также верна.
\end{proofnote}

\begin{theorem}[Производная обратной функции]
Если $f$ непрерывна и строго монотонна на промежутке, дифференцируема в точке $x$ и $f'(x) \neq 0$, то обратная функция $f^{-1}$ дифференцируема в точке $y = f(x)$, и:
\[
(f^{-1})'(y) = \frac{1}{f'(x)} = \frac{1}{f'(f^{-1}(y))}.
\]
\end{theorem}

\begin{proofnote}
Пусть $y = f(x)$, $y + k = f(x + h)$. Тогда $h = f^{-1}(y+k) - f^{-1}(y)$.
\[
\frac{f^{-1}(y+k) - f^{-1}(y)}{k} = \frac{h}{f(x+h) - f(x)} \to \frac{1}{f'(x)}.
\]
\end{proofnote}

\section*{Билет 56. Первая теорема Вейерштрасса}

\begin{theorem}[Первая теорема Вейерштрасса]
Если функция $f$ непрерывна на отрезке $[a, b]$, то она ограничена на $[a, b]$.
\end{theorem}

\begin{proofnote}
От противного. Пусть $f$ не ограничена.
Тогда для каждого $n \in \N$ найдётся $x_n \in [a, b]$ такое, что $|f(x_n)| > n$.
Последовательность $\{x_n\}$ ограничена, по теореме Больцано–Вейерштрасса выделяем сходящуюся подпоследовательность $x_{n_k} \to c \in [a, b]$.
По непрерывности $f$ в точке $c$: $f(x_{n_k}) \to f(c)$, но $|f(x_{n_k})| > n_k \to \infty$ — противоречие.
\end{proofnote}

\section*{Билет 57. Вторая теорема Вейерштрасса}

\begin{theorem}[Вторая теорема Вейерштрасса]
Если функция $f$ непрерывна на отрезке $[a, b]$, то она достигает своих наибольшего и наименьшего значений на $[a, b]$.
\end{theorem}

\begin{proofnote}
По первой теореме Вейерштрасса $f$ ограничена. Пусть $M = \sup\limits_{[a,b]} f(x)$.
Тогда найдётся последовательность $\{x_n\} \subset [a, b]$ такая, что $f(x_n) \to M$.
Выделим сходящуюся подпоследовательность $x_{n_k} \to c \in [a, b]$.
По непрерывности $f(c) = \lim f(x_{n_k}) = M$.
Аналогично для минимума.
\end{proofnote}

\section*{Билет 58. Теорема Ферма}

\begin{theorem}[Ферма]
Если функция $f$ имеет локальный экстремум в точке $x_0$ (внутренней точке области определения) и дифференцируема в этой точке, то $f'(x_0) = 0$.
\end{theorem}

\begin{proofnote}
Пусть $x_0$ — точка локального максимума.
Тогда для малых $h > 0$: $f(x_0 + h) \leq f(x_0)$, откуда:
\[
\frac{f(x_0 + h) - f(x_0)}{h} \leq 0 \Rightarrow f'_+(x_0) \leq 0.
\]
Для $h < 0$: $f(x_0 + h) \leq f(x_0)$, откуда:
\[
\frac{f(x_0 + h) - f(x_0)}{h} \geq 0 \Rightarrow f'_-(x_0) \geq 0.
\]
Так как $f$ дифференцируема, $f'_+(x_0) = f'_-(x_0) = f'(x_0)$, следовательно, $f'(x_0) = 0$.
\end{proofnote}

\section*{Билет 59. Теорема Ролля}

\begin{theorem}[Ролль]
Если функция $f$ непрерывна на $[a, b]$, дифференцируема на $(a, b)$ и $f(a) = f(b)$, то найдётся точка $c \in (a, b)$ такая, что $f'(c) = 0$.
\end{theorem}

\begin{proofnote}
По теореме Вейерштрасса $f$ достигает на $[a, b]$ наибольшего $M$ и наименьшего $m$ значений.
Если $M = m$, то $f$ постоянна, и $f'(x) = 0$ для всех $x$.
Если $M > m$, то хотя бы одно из этих значений достигается во внутренней точке $c \in (a, b)$.
По теореме Ферма $f'(c) = 0$.
\end{proofnote}

\section*{Билет 60. Теорема Лагранжа}

\begin{theorem}[Лагранж]
Если функция $f$ непрерывна на $[a, b]$ и дифференцируема на $(a, b)$, то найдётся точка $c \in (a, b)$ такая, что:
\[
f(b) - f(a) = f'(c)(b - a).
\]
\end{theorem}

\begin{proofnote}
Рассмотрим функцию:
\[
F(x) = f(x) - f(a) - \frac{f(b) - f(a)}{b - a} (x - a).
\]
Тогда $F(a) = F(b) = 0$.
По теореме Ролля найдётся $c \in (a, b)$: $F'(c) = 0$, т.е.:
\[
f'(c) - \frac{f(b) - f(a)}{b - a} = 0.
\]
\end{proofnote}

\section*{Билет 61. Теорема Коши}

\begin{theorem}[Коши]
Если функции $f$ и $g$ непрерывны на $[a, b]$, дифференцируемы на $(a, b)$ и $g'(x) \neq 0$ для всех $x \in (a, b)$, то найдётся точка $c \in (a, b)$ такая, что:
\[
\frac{f(b) - f(a)}{g(b) - g(a)} = \frac{f'(c)}{g'(c)}.
\]
\end{theorem}

\begin{proofnote}
Заметим, что $g(b) \neq g(a)$ (иначе по теореме Ролля $g'(x) = 0$ для некоторого $x$).
Рассмотрим функцию:
\[
F(x) = f(x) - f(a) - \frac{f(b) - f(a)}{g(b) - g(a)} (g(x) - g(a)).
\]
Тогда $F(a) = F(b) = 0$. По теореме Ролля найдётся $c \in (a, b)$: $F'(c) = 0$, т.е.:
\[
f'(c) - \frac{f(b) - f(a)}{g(b) - g(a)} g'(c) = 0.
\]
\end{proofnote}

\section*{Билет 62. $c'$, $x'$, $(x^n)'$, $(1/x^n)'$}

\begin{theorem}
\begin{enumerate}
    \item $(c)' = 0$.
    \item $(x)' = 1$.
    \item $(x^n)' = n x^{n-1}$ для $n \in \N$.
    \item $\left( \frac{1}{x^n} \right)' = -\frac{n}{x^{n+1}}$ для $n \in \N$.
\end{enumerate}
\end{theorem}

\begin{proofnote}
\begin{enumerate}
    \item По определению производной.
    \item $\frac{(x+h) - x}{h} = 1$.
    \item По индукции или через бином Ньютона.
    \item Применяем правило для $(x^{-n})' = -n x^{-n-1}$.
\end{enumerate}
\end{proofnote}

\section*{Билет 63. $(e^x)'$, $(\ln x)'$}

\begin{theorem}
\begin{enumerate}
    \item $(e^x)' = e^x$.
    \item $(\ln x)' = \frac{1}{x}$ для $x > 0$.
\end{enumerate}
\end{theorem}

\begin{proofnote}
\begin{enumerate}
    \item \[
    \frac{e^{x+h} - e^x}{h} = e^x \cdot \frac{e^h - 1}{h} \to e^x \cdot 1 = e^x.
    \]
    \item \[
    \frac{\ln(x+h) - \ln x}{h} = \frac{1}{h} \ln\left(1 + \frac{h}{x}\right) = \frac{\ln\left(1 + \frac{h}{x}\right)}{\frac{h}{x}} \cdot \frac{1}{x} \to 1 \cdot \frac{1}{x} = \frac{1}{x}.
    \]
\end{enumerate}
\end{proofnote}

\section*{Билет 64. $(x^r)'$}

\begin{theorem}
Для любого $r \in \R$:
\[
(x^r)' = r x^{r-1}, \quad x > 0.
\]
\end{theorem}

\begin{proofnote}
$x^r = e^{r \ln x}$. По правилу дифференцирования композиции:
\[
(x^r)' = e^{r \ln x} \cdot \frac{r}{x} = x^r \cdot \frac{r}{x} = r x^{r-1}.
\]
\end{proofnote}

\section*{Билет 65. $(\sin x)'$, $(\cos x)'$}

\begin{theorem}
\begin{enumerate}
    \item $(\sin x)' = \cos x$.
    \item $(\cos x)' = -\sin x$.
\end{enumerate}
\end{theorem}

\begin{proofnote}
\begin{enumerate}
    \item \[
    \frac{\sin(x+h) - \sin x}{h} = \frac{2 \sin\frac{h}{2} \cos\left(x + \frac{h}{2}\right)}{h} = \frac{\sin\frac{h}{2}}{\frac{h}{2}} \cdot \cos\left(x + \frac{h}{2}\right) \to 1 \cdot \cos x.
    \]
    \item Аналогично или через $\cos x = \sin\left(\frac{\pi}{2} - x\right)$.
\end{enumerate}
\end{proofnote}

\section*{Билет 66. $(\tg x)'$, $(\ctg x)'$}

\begin{theorem}
\begin{enumerate}
    \item $(\tg x)' = \frac{1}{\cos^2 x} = \sec^2 x$.
    \item $(\ctg x)' = -\frac{1}{\sin^2 x} = -\csc^2 x$.
\end{enumerate}
\end{theorem}

\begin{proofnote}
\begin{enumerate}
    \item $\tg x = \frac{\sin x}{\cos x}$. По правилу частного:
    \[
    (\tg x)' = \frac{\cos x \cdot \cos x - \sin x \cdot (-\sin x)}{\cos^2 x} = \frac{\cos^2 x + \sin^2 x}{\cos^2 x} = \frac{1}{\cos^2 x}.
    \]
    \item Аналогично для $\ctg x = \frac{\cos x}{\sin x}$.
\end{enumerate}
\end{proofnote}

\section*{Билет 67. $(\arcsin x)'$, $(\arccos x)'$}

\begin{theorem}
\begin{enumerate}
    \item $(\arcsin x)' = \frac{1}{\sqrt{1 - x^2}}$ для $|x| < 1$.
    \item $(\arccos x)' = -\frac{1}{\sqrt{1 - x^2}}$ для $|x| < 1$.
\end{enumerate}
\end{theorem}

\begin{proofnote}
Для $\arcsin x$: пусть $y = \arcsin x$, тогда $x = \sin y$, $-\frac{\pi}{2} < y < \frac{\pi}{2}$.
По теореме о производной обратной функции:
\[
(\arcsin x)' = \frac{1}{(\sin y)'} = \frac{1}{\cos y} = \frac{1}{\sqrt{1 - \sin^2 y}} = \frac{1}{\sqrt{1 - x^2}}.
\]
Для $\arccos x$ аналогично или через $\arccos x = \frac{\pi}{2} - \arcsin x$.
\end{proofnote}

\section*{Билет 68. $(\arctg x)'$, $(\arcctg x)'$}

\begin{theorem}
\begin{enumerate}
    \item $(\arctg x)' = \frac{1}{1 + x^2}$.
    \item $(\arcctg x)' = -\frac{1}{1 + x^2}$.
\end{enumerate}
\end{theorem}

\begin{proofnote}
Для $\arctg x$: пусть $y = \arctg x$, тогда $x = \tg y$, $-\frac{\pi}{2} < y < \frac{\pi}{2}$.
\[
(\arctg x)' = \frac{1}{(\tg y)'} = \frac{1}{\frac{1}{\cos^2 y}} = \cos^2 y = \frac{1}{1 + \tg^2 y} = \frac{1}{1 + x^2}.
\]
Для $\arcctg x$ аналогично или через $\arcctg x = \frac{\pi}{2} - \arctg x$.
\end{proofnote}

\section*{Билет 69. Определение $f^{(n)}(x)$ производной n-го порядка}

\begin{definition}
Производная нулевого порядка: $f^{(0)}(x) = f(x)$.
Производная первого порядка: $f^{(1)}(x) = f'(x)$.
Производная $n$-го порядка ($n \geq 2$):
\[
f^{(n)}(x) = \left( f^{(n-1)}(x) \right)'.
\]
\end{definition}

\section*{Билет 70. $(cf)^{(n)}$, $(f + g)^{(n)}$}

\begin{theorem}
\begin{enumerate}
    \item $(cf)^{(n)} = c f^{(n)}$.
    \item $(f + g)^{(n)} = f^{(n)} + g^{(n)}$.
\end{enumerate}
\end{theorem}

\begin{proofnote}
По индукции, используя линейность первой производной.
\end{proofnote}

\section*{Билет 71. $(f(ax+b))^{(n)}$}

\begin{theorem}
Если $g(x) = f(ax + b)$, то:
\[
g^{(n)}(x) = a^n f^{(n)}(ax + b).
\]
\end{theorem}

\begin{proofnote}
Индукция по $n$.
База $n=1$: $g'(x) = a f'(ax + b)$.
Шаг: предположим верно для $n$, тогда:
\[
g^{(n+1)}(x) = \left( a^n f^{(n)}(ax + b) \right)' = a^n \cdot a f^{(n+1)}(ax + b) = a^{n+1} f^{(n+1)}(ax + b).
\]
\end{proofnote}

\section*{Билет 72. $(x^k)^{(n)}$}

\begin{theorem}
Для натурального $k$:
\[
(x^k)^{(n)} = 
\begin{cases}
\frac{k!}{(k-n)!} x^{k-n}, & n \leq k, \\
0, & n > k.
\end{cases}
\]
\end{theorem}

\begin{proofnote}
Индукция по $n$.
Для $n=1$: $(x^k)' = k x^{k-1}$.
Предположим верно для $n$: $(x^k)^{(n)} = k(k-1)\dots(k-n+1) x^{k-n}$.
Тогда:
\[
(x^k)^{(n+1)} = \left( k(k-1)\dots(k-n+1) x^{k-n} \right)' = k(k-1)\dots(k-n+1)(k-n) x^{k-n-1}.
\]
При $n = k$: $(x^k)^{(k)} = k!$. При $n > k$: производная равна нулю.
\end{proofnote}

\section*{Билет 73. $(x^r)^{(n)}$}

\begin{theorem}
Для $r \in \R$, $x > 0$:
\[
(x^r)^{(n)} = r(r-1)\dots(r-n+1) x^{r-n}.
\]
\end{theorem}

\begin{proofnote}
Индукция по $n$.
База: $(x^r)' = r x^{r-1}$.
Шаг: аналогично билету 72.
\end{proofnote}

\section*{Билет 74. $(e^x)^{(n)}$, $(\ln x)^{(n)}$}

\begin{theorem}
\begin{enumerate}
    \item $(e^x)^{(n)} = e^x$.
    \item $(\ln x)^{(n)} = (-1)^{n-1} \frac{(n-1)!}{x^n}$ для $n \geq 1$.
\end{enumerate}
\end{theorem}

\begin{proofnote}
\begin{enumerate}
    \item Очевидно по индукции.
    \item База: $(\ln x)' = \frac{1}{x}$.
    Предположим: $(\ln x)^{(n)} = (-1)^{n-1} \frac{(n-1)!}{x^n}$.
    Тогда:
    \[
    (\ln x)^{(n+1)} = \left( (-1)^{n-1} (n-1)! x^{-n} \right)' = (-1)^{n-1} (n-1)! (-n) x^{-n-1} = (-1)^n n! x^{-(n+1)}.
    \]
\end{enumerate}
\end{proofnote}

\section*{Билет 75. $(\sin x)^{(n)}$, $(\cos x)^{(n)}$}

\begin{theorem}
\begin{enumerate}
    \item $(\sin x)^{(n)} = \sin\left( x + \frac{n\pi}{2} \right)$.
    \item $(\cos x)^{(n)} = \cos\left( x + \frac{n\pi}{2} \right)$.
\end{enumerate}
\end{theorem}

\begin{proofnote}
Индукция по $n$.
База: $(\sin x)' = \cos x = \sin(x + \pi/2)$.
Шаг: предположим верно для $n$, тогда:
\[
(\sin x)^{(n+1)} = \left( \sin\left( x + \frac{n\pi}{2} \right) \right)' = \cos\left( x + \frac{n\pi}{2} \right) = \sin\left( x + \frac{(n+1)\pi}{2} \right).
\]
Аналогично для $\cos x$.
\end{proofnote}

\section*{Билет 76. $((x-a)^k)^{(n)}$, $((1+x)^r)^{(n)}$, $(\ln(1+x))^{(n)}$}

\begin{theorem}
\begin{enumerate}
    \item $((x-a)^k)^{(n)} = \frac{k!}{(k-n)!} (x-a)^{k-n}$ при $n \leq k$, иначе 0.
    \item $((1+x)^r)^{(n)} = r(r-1)\dots(r-n+1) (1+x)^{r-n}$.
    \item $(\ln(1+x))^{(n)} = (-1)^{n-1} \frac{(n-1)!}{(1+x)^n}$ при $n \geq 1$.
\end{enumerate}
\end{theorem}

\begin{proofnote}
Следует из билетов 72, 73, 74 с соответствующими сдвигами.
\end{proofnote}

\section*{Билет 77. Формула Тейлора для полинома}

\begin{theorem}
Если $P(x)$ — полином степени $n$:
\[
P(x) = \sum_{k=0}^n a_k x^k,
\]
то для любого $a$:
\[
P(x) = \sum_{k=0}^n \frac{P^{(k)}(a)}{k!} (x - a)^k.
\]
\end{theorem}

\begin{proofnote}
Подставим $x = a + t$, раскроем $P(a+t)$ по степеням $t$ и найдём коэффициенты через производные в точке $a$.
\end{proofnote}

\section*{Билет 78. Формула Тейлора с остатком в форме Пеано}

\begin{theorem}[Тейлор–Пеано]
Если функция $f$ имеет в точке $a$ производные до порядка $n$, то:
\[
f(x) = \sum_{k=0}^n \frac{f^{(k)}(a)}{k!} (x - a)^k + o((x-a)^n), \quad x \to a.
\]
\end{theorem}

\begin{proofnote}
Индукция по $n$.
База $n=1$: это определение дифференцируемости.
Шаг: используем правило Лопиталя или представление через интеграл.
\end{proofnote}

\section*{Билет 79. Формула Тейлора с остатком в форме Лагранжа}

\begin{theorem}[Тейлор–Лагранж]
Если $f$ имеет на $[a, x]$ непрерывные производные до порядка $n$, а на $(a, x)$ — производную порядка $n+1$, то найдётся точка $c \in (a, x)$ такая, что:
\[
f(x) = \sum_{k=0}^n \frac{f^{(k)}(a)}{k!} (x - a)^k + \frac{f^{(n+1)}(c)}{(n+1)!} (x - a)^{n+1}.
\]
\end{theorem}

\begin{proofnote}
Рассмотрим функцию:
\[
R_n(t) = f(x) - \sum_{k=0}^n \frac{f^{(k)}(t)}{k!} (x - t)^k.
\]
Покажем, что $R_n(a) = \frac{f^{(n+1)}(c)}{(n+1)!} (x-a)^{n+1}$, применяя теорему Коши к $R_n(t)$ и $(x-t)^{n+1}$.
\end{proofnote}

\section*{Билет 80. Применение формулы Тейлора к $e^x$}

\begin{theorem}
Для любого $n$:
\[
e^x = \sum_{k=0}^n \frac{x^k}{k!} + \frac{e^c}{(n+1)!} x^{n+1}, \quad c \in (0, x) \text{ или } (x, 0).
\]
\end{theorem}

\begin{proofnote}
По формуле Тейлора–Лагранжа с $a=0$, $f^{(k)}(0) = 1$ для всех $k$.
\end{proofnote}

\section*{Билет 81. Применение формулы Тейлора к $(1+x)^r$}

\begin{theorem}
Для $r \in \R$:
\[
(1+x)^r = \sum_{k=0}^n \binom{r}{k} x^k + \binom{r}{n+1} (1+c)^{r-n-1} x^{n+1},
\]
где $\binom{r}{k} = \frac{r(r-1)\dots(r-k+1)}{k!}$, $c$ между 0 и $x$.
\end{theorem}

\begin{proofnote}
По формуле Тейлора–Лагранжа с $a=0$, $f^{(k)}(0) = r(r-1)\dots(r-k+1)$.
\end{proofnote}

\section*{Билет 82. Применение формулы Тейлора к $\ln(1+x)$}

\begin{theorem}
Для $x > -1$:
\[
\ln(1+x) = \sum_{k=1}^n (-1)^{k-1} \frac{x^k}{k} + (-1)^n \frac{x^{n+1}}{(n+1)(1+c)^{n+1}},
\]
где $c$ между 0 и $x$.
\end{theorem}

\begin{proofnote}
По формуле Тейлора–Лагранжа с $a=0$, $f^{(k)}(0) = (-1)^{k-1} (k-1)!$.
\end{proofnote}

\section*{Билет 83. Применение формулы Тейлора к $\sin x$, $\cos x$}

\begin{theorem}
\begin{enumerate}
    \item \[
    \sin x = \sum_{k=0}^n (-1)^k \frac{x^{2k+1}}{(2k+1)!} + (-1)^{n+1} \frac{\cos c}{(2n+3)!} x^{2n+3}.
    \]  
    \item \[
    \cos x = \sum_{k=0}^n (-1)^k \frac{x^{2k}}{(2k)!} + (-1)^{n+1} \frac{\cos c}{(2n+2)!} x^{2n+2}.
    \]
\end{enumerate}
\end{theorem}

\begin{proofnote}
По формуле Тейлора–Лагранжа с $a=0$, учитывая, что производные синуса и косинуса периодичны.
\end{proofnote}

\end{document}
